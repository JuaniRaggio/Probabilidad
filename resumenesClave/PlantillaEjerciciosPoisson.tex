\documentclass{article}
\usepackage{amsmath}
\usepackage{amssymb}
\usepackage{geometry}
\geometry{a4paper, margin=1in}

\begin{document}

\section{Plantilla de resolucion --- Procesos de Poisson (TP6)}

\subsection{Definicion general}

Un proceso de Poisson es un proceso estocastico que modela la ocurrencia de eventos aleatorios en el tiempo.

\begin{itemize}
  \item Tasa de ocurrencia: $\lambda$ (eventos por unidad de tiempo).
  \item Variable de conteo: $N(t)$: cantidad de eventos hasta el tiempo $t$.
  \item Espacio de estados: $\{0, 1, 2, 3, \dots\}$.
\end{itemize}

\subsection*{Propiedades fundamentales}

\begin{itemize}
  \item \textbf{Incrementos independientes:} los eventos en intervalos disjuntos son independientes.
  \item \textbf{Incrementos estacionarios:} la distribucion depende solo de la longitud del intervalo.
  \item \textbf{Aproximacion en intervalos pequenos:}
  \[
  P(\text{1 evento en } \Delta t) \approx \lambda \Delta t
  \]
  \[
  P(\text{2 o mas eventos en } \Delta t) \approx 0
  \]
\end{itemize}

\subsection*{Distribuci\'on de $N(t)$}

La cantidad de eventos en un intervalo de longitud $t$ sigue una distribuci\'on de Poisson:

\[
P(N(t) = k) = \frac{(\lambda t)^k}{k!} e^{-\lambda t}
\]

o bien:

\[
N(t) \sim \text{Poisson}(\lambda t)
\]

\subsection*{Distribucion de los tiempos entre eventos}

Los tiempos entre eventos consecutivos ($T_i$) son variables aleatorias independientes con distribucion exponencial:

\[
T_i \sim \text{Exponencial}(\lambda)
\]

\subsection*{Esperanza y varianza}

\begin{itemize}
  \item Esperanza: $E[N(t)] = \lambda t$
  \item Varianza: $\text{Var}[N(t)] = \lambda t$
\end{itemize}

\end{document}
